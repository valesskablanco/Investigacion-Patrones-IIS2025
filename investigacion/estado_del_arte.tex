\section{Estado del arte}

\label{sec:intro}

Entre las técnicas más conocidas se encuentran los métodos de \textit{clustering} o agrupamiento. Uno de los más representativos es \textit{K-Means}, propuesto por MacQueen \cite{macqueen1967some}, el cual divide los datos en grupos según su similitud. Otro método importante es \textit{DBSCAN}, que detecta regiones densas de puntos sin necesidad de definir previamente el número de grupos. Asimismo, los métodos de reducción de dimensionalidad, como el \textit{Análisis de Componentes Principales} (PCA) y el \textit{t-Distributed Stochastic Neighbor Embedding} (t-SNE), permiten representar los datos en espacios de menor dimensión, facilitando su visualización e interpretación \cite{Hinton2006, vandermaaten2008tsne}.

El desarrollo de las redes neuronales profundas ha impulsado nuevas técnicas no supervisadas más potentes, como los \textit{autoencoders} y las \textit{Generative Adversarial Networks} (GANs). Los autoencoders aprenden a codificar y reconstruir los datos de entrada, logrando representar sus características esenciales \cite{Hinton2006}. Por su parte, las GANs, introducidas por Goodfellow et al. \cite{goodfellow2014gan}, emplean una red generadora y una discriminadora que compiten entre sí, produciendo datos sintéticos de alta calidad. Estos avances han sido ampliamente documentados en la literatura de aprendizaje profundo \cite{lecun2015deep, Goodfellow2016}.

Además, han surgido enfoques híbridos como el \textit{Unsupervised Deep Embedding} (DEC), el cual busca optimizar simultáneamente la representación latente y la asignación de grupos en los datos \cite{xie2016unsupervised}. Dichos modelos se aplican hoy en día en campos como la segmentación de clientes, la detección de anomalías, el análisis de texto y la clasificación de imágenes médicas.

A pesar de los avances logrados, el aprendizaje no supervisado todavía enfrenta desafíos importantes, entre ellos la dificultad para evaluar los resultados sin etiquetas de referencia, la sensibilidad a los parámetros iniciales y la interpretación de los patrones descubiertos. Actualmente, la investigación avanza hacia el aprendizaje auto-supervisado, una rama emergente que busca generar etiquetas de manera automática a partir de los propios datos, reduciendo la dependencia del etiquetado manual y combinando lo mejor de los enfoques supervisados y no supervisados \cite{lecun2015deep, Goodfellow2016}.

% Contenido de la Introducción
\section{Introducción}
\label{sec:intro}

El aprendizaje automático se ha consolidado como una herramienta fundamental en la extracción de conocimiento a partir de grandes volúmenes de datos. Tradicionalmente, muchos de los métodos más conocidos se enmarcan en el aprendizaje supervisado, donde un algoritmo aprende a partir de un conjunto de datos previamente etiquetado, es decir, donde cada ejemplo de entrada está asociado con una salida o respuesta correcta. Sin embargo, en numerosos escenarios del mundo real, la obtención de estas etiquetas es un proceso costoso, lento o, en ocasiones, simplemente imposible. Es en este contexto donde el aprendizaje no supervisado adquiere una relevancia crítica \cite{Bishop2006}

El aprendizaje no supervisado (Unsupervised Learning) aborda el desafío de encontrar patrones, estructuras o conocimiento inherente en conjuntos de datos que carecen por completo de etiquetas. A diferencia de su contraparte supervisada, el objetivo no es predecir una salida específica, sino más bien explorar la estructura intrínseca de los datos. Los algoritmos de este paradigma actúan como exploradores, buscando afinidades, anomalías o representaciones simplificadas sin ninguna guía externa sobre lo que constituye un resultado "correcto". \cite{Goodfellow2016}

Las tareas principales del aprendizaje no supervisado suelen dividirse en dos grandes familias: la clusterización (agrupamiento) y la reducción de dimensionalidad. La clusterización busca agrupar los datos en "clústeres" o grupos, de tal manera que los puntos dentro de un mismo grupo sean muy similares entre sí y muy diferentes a los puntos de otros grupos. Por otro lado, la reducción de dimensionalidad se enfoca en simplificar los datos, reduciendo el número de variables (características) a considerar, pero preservando la información más relevante. Este proceso no solo facilita la visualización, sino que también optimiza el almacenamiento y el rendimiento de otros algoritmos de aprendizaje.\cite{Hinton2006}
% --- USA LA CLASE IEEEtran ---
% "conference" lo pone a dos columnas, que es lo más común.
% Para formato de revista (journal), usa [journal]
\documentclass[conference]{IEEEtran}

% --- PAQUETES COMUNES EN IEEE ---
\usepackage[utf8]{inputenc} % Acentos y caracteres
\usepackage[T1]{fontenc}    % Para una mejor codificación de fuentes
\usepackage{graphicx}       % Para imágenes
\usepackage{amsmath}        % Para matemáticas avanzadas
\usepackage{cite}           % Para formatear las citas (ej. [1]-[3])
\usepackage{url}            % Para URLs

\begin{document}
% --- INFORMACIÓN DEL DOCUMENTO ---
\title{Investigacion Patrones (No me acuerdo del nombre)}

% --- ESTRUCTURA DE AUTORES (MUY DIFERENTE) ---
% Usa \IEEEauthorblockN para nombres y \IEEEauthorblockA para afiliación
\author{
    \IEEEauthorblockN{Jose Eduardo Cruz Vargas\\
    Darío Rámses Gutierréz Rodríguez\\
    Valeska Blanco\\
    Andres Martinez Vargas}
    \IEEEauthorblockA{
        Instituto Tecnologico de Costa Rica\\
        Cartago, Costa Rica\\
        Email: jecruz@estudiantec.cr
    }
    % Descomenta y repite este bloque para más autores
    % \and
    % \IEEEauthorblockN{Nombre del Segundo Autor}
    % \IEEEauthorblockA{
    %     Afiliación del Segundo Autor\\
    %     Ciudad, País\\
    %     Email: otro.email@dominio.com
    % }
}

% --- INICIO DEL DOCUMENTO ---


% --- RESUMEN (ABSTRACT) ---
% El abstract DEBE ir antes de \maketitle en IEEEtran



% --- CREA EL TÍTULO Y EL BLOQUE DE AUTORES ---
% \maketitle DEBE ir DESPUÉS del abstract en IEEEtran
\maketitle


% --- CONTENIDO DEL DOCUMENTO ---
% Tu archivo \input funciona exactamente igual
\begin{abstract}
Este es el resumen (abstract) de tu documento. Debe ser conciso
y describir el propósito, la metodología y los hallazgos 
principales de tu trabajo.
\end{abstract}

% Opcional: Palabras Clave
\begin{IEEEkeywords}
Palabra clave 1, palabra clave 2, LaTeX, IEEE.
\end{IEEEkeywords}
% Contenido de la Introducción
\section{Introducción}
\label{sec:intro}

El aprendizaje automático se ha consolidado como una herramienta fundamental en la extracción de conocimiento a partir de grandes volúmenes de datos. Tradicionalmente, muchos de los métodos más conocidos se enmarcan en el aprendizaje supervisado, donde un algoritmo aprende a partir de un conjunto de datos previamente etiquetado, es decir, donde cada ejemplo de entrada está asociado con una salida o respuesta correcta. Sin embargo, en numerosos escenarios del mundo real, la obtención de estas etiquetas es un proceso costoso, lento o, en ocasiones, simplemente imposible. Es en este contexto donde el aprendizaje no supervisado adquiere una relevancia crítica \cite{Bishop2006}

El aprendizaje no supervisado (Unsupervised Learning) aborda el desafío de encontrar patrones, estructuras o conocimiento inherente en conjuntos de datos que carecen por completo de etiquetas. A diferencia de su contraparte supervisada, el objetivo no es predecir una salida específica, sino más bien explorar la estructura intrínseca de los datos. Los algoritmos de este paradigma actúan como exploradores, buscando afinidades, anomalías o representaciones simplificadas sin ninguna guía externa sobre lo que constituye un resultado "correcto". \cite{Goodfellow2016}

Las tareas principales del aprendizaje no supervisado suelen dividirse en dos grandes familias: la clusterización (agrupamiento) y la reducción de dimensionalidad. La clusterización busca agrupar los datos en "clústeres" o grupos, de tal manera que los puntos dentro de un mismo grupo sean muy similares entre sí y muy diferentes a los puntos de otros grupos. Por otro lado, la reducción de dimensionalidad se enfoca en simplificar los datos, reduciendo el número de variables (características) a considerar, pero preservando la información más relevante. Este proceso no solo facilita la visualización, sino que también optimiza el almacenamiento y el rendimiento de otros algoritmos de aprendizaje.\cite{Hinton2006}
\section{Desarrollo}
\label{sec:intro}

Para el desarrollo del caso práctico se decidió elaborar un autoencoder con sobre el conjunto de datos de MNIST. Para iniciar se importaron las librerías de PyTorch, Matplotlib y NumPy. Específicamente, PyTorch fue la biblioteca central: se usó \texttt{torch} para las operaciones de tensores y la configuración del dispositivo; \texttt{torch.nn} para construir la clase, sus capas y la función de pérdida, además \texttt{torch.optim} para el optimizador. Para la gestión de los datos, se importaron \texttt{DataLoader} de PyTorch y los módulos \texttt{datasets} y \texttt{transforms} de Torchvision para cargar y pre-procesar el dataset MNIST. Complementariamente, se incluyó \texttt{matplotlib.pyplot} para todas las visualizaciones y \texttt{numpy} para las operaciones numéricas de preparación de datos para el análisis del espacio latente.

A continuación, se definieron los hiperparámetros cruciales para el entrenamiento. Estos valores controlan el proceso de aprendizaje: el \texttt{batch\_size} (tamaño de lote) se fijó en 128, indicando cuántas imágenes se procesan juntas antes de actualizar el modelo; la \texttt{learning\_rate} (tasa de aprendizaje) se estableció en 0.001 ($1e-3$), controlando el tamaño del paso que da el optimizador Adam para ajustar los pesos de la red; el \texttt{num\_epochs} se fijó en 20, determinando cuántas veces el modelo vería el conjunto de datos de entrenamiento completo; y finalmente, el \texttt{latent\_dim} (dimensión latente) se estableció en 32, definiendo el tamaño del "cuello de botella", que es la representación comprimida de la imagen.

En el paso siguiente, se procedió a la preparación de los datos. Se utilizó la biblioteca \texttt{torchvision.datasets} para descargar y cargar automáticamente el dataset MNIST, el cual ya venía previamente definido con su división estándar de 60,000 imágenes para el entrenamiento (\texttt{train=True}) y 10,000 imágenes para las pruebas (\texttt{train=False}). Esta partición dedica aproximadamente el 85.7\% del conjunto al entrenamiento y el 14.3\% restante al testeo. Fue crucial aplicar transformaciones a cada imagen: primero se convirtieron a tensores de PyTorch y luego se normalizaron con media 0.5 y desviación 0.5 para adecuar los valores de los píxeles. Finalmente, estos datasets se envolvieron en \texttt{DataLoaders} para suministrar los datos al modelo de forma eficiente en lotes de 128 imágenes.

Posteriormente, se definió la arquitectura mediante la clase \texttt{Autoencoder}. Esta clase encapsula dos componentes principales: el \texttt{Encoder} (codificador), una red secuencial que reduce la dimensionalidad de la imagen de entrada (784 píxeles) al espacio latente de 32 dimensiones; y el \texttt{Decoder} (decodificador), que revierte este proceso intentando reconstruir la imagen original a partir de esa representación compacta. Sumado a esto, se implementó la función \texttt{forward}, esta función recibe el tensor de entrada (\texttt{x}), lo pasa secuencialmente a través del \texttt{encoder} para obtener el vector latente y luego pasa ese vector latente por el \texttt{decoder} para generar la imagen reconstruida, la cual se retorna. Además, se configuraron los componentes del entrenamiento: se seleccionó el Error Cuadrático Medio como la función de pérdida para medir la diferencia entre la imagen original y la reconstruida, y se instanció el optimizador Adam para manejar la actualización de los pesos del modelo.

Para finalizar, se entrenó el modelo del autoencoder durante 20 épocas. En cada época, primero se ajustaban los pesos del modelo usando los datos de entrenamiento, calculando la pérdida (error) entre las imágenes originales y las reconstruidas mediante MSE y aplicando retropropagación. Inmediatamente después, se evaluaba el rendimiento del modelo con los datos de prueba, calculando la pérdida de la misma maner, y se guardaban los resultados de ambas fases.

Al finalizar el entrenamiento, se graficaron las curvas de pérdida de entrenamiento y prueba para observar la evolución del aprendizaje. Luego, se visualizó el desempeño del modelo tomando un lote de imágenes de prueba: se mostraron las imágenes originales en una fila y, justo debajo, sus correspondientes imágenes reconstruidas por el autoencoder. Finalmente, se utilizó la técnica t-SNE para comprimir los vectores latentes (la salida del \texttt{encoder}) de todo el conjunto de prueba a solo dos dimensiones, creando un gráfico de dispersión coloreado por dígito para visualizar cómo el modelo agrupó las imágenes.

\section{Estado del arte}

\label{sec:intro}

Entre las técnicas más conocidas se encuentran los métodos de \textit{clustering} o agrupamiento. Uno de los más representativos es \textit{K-Means}, propuesto por MacQueen \cite{macqueen1967some}, el cual divide los datos en grupos según su similitud. Otro método importante es \textit{DBSCAN}, que detecta regiones densas de puntos sin necesidad de definir previamente el número de grupos. Asimismo, los métodos de reducción de dimensionalidad, como el \textit{Análisis de Componentes Principales} (PCA) y el \textit{t-Distributed Stochastic Neighbor Embedding} (t-SNE), permiten representar los datos en espacios de menor dimensión, facilitando su visualización e interpretación \cite{Hinton2006, vandermaaten2008tsne}.

El desarrollo de las redes neuronales profundas ha impulsado nuevas técnicas no supervisadas más potentes, como los \textit{autoencoders} y las \textit{Generative Adversarial Networks} (GANs). Los autoencoders aprenden a codificar y reconstruir los datos de entrada, logrando representar sus características esenciales \cite{Hinton2006}. Por su parte, las GANs, introducidas por Goodfellow et al. \cite{goodfellow2014gan}, emplean una red generadora y una discriminadora que compiten entre sí, produciendo datos sintéticos de alta calidad. Estos avances han sido ampliamente documentados en la literatura de aprendizaje profundo \cite{lecun2015deep, Goodfellow2016}.

Además, han surgido enfoques híbridos como el \textit{Unsupervised Deep Embedding} (DEC), el cual busca optimizar simultáneamente la representación latente y la asignación de grupos en los datos \cite{xie2016unsupervised}. Dichos modelos se aplican hoy en día en campos como la segmentación de clientes, la detección de anomalías, el análisis de texto y la clasificación de imágenes médicas.

A pesar de los avances logrados, el aprendizaje no supervisado todavía enfrenta desafíos importantes, entre ellos la dificultad para evaluar los resultados sin etiquetas de referencia, la sensibilidad a los parámetros iniciales y la interpretación de los patrones descubiertos. Actualmente, la investigación avanza hacia el aprendizaje auto-supervisado, una rama emergente que busca generar etiquetas de manera automática a partir de los propios datos, reduciendo la dependencia del etiquetado manual y combinando lo mejor de los enfoques supervisados y no supervisados \cite{lecun2015deep, Goodfellow2016}.

\section{Resultados}
\label{sec:intro}
\section{Conclusiones}
\label{sec:conclusiones}

El presente trabajo permitió evaluar el desempeño de un modelo autoencoder aplicado al conjunto de datos \textit{MNIST}, demostrando la efectividad del aprendizaje no supervisado para la extracción de características relevantes sin la necesidad de etiquetas.

En primer lugar, el análisis de las curvas de pérdida evidenció un proceso de entrenamiento estable y progresivo. La disminución consistente del error en los conjuntos de entrenamiento y de prueba indicó que el modelo logró aprender representaciones útiles sin incurrir en sobreajuste. Este comportamiento sugiere que la arquitectura y los hiperparámetros seleccionados fueron adecuados para la complejidad del problema.

En segundo lugar, las reconstrucciones generadas por el autoencoder mostraron un alto grado de similitud con las imágenes originales. Las formas, proporciones y trazos característicos de los dígitos fueron preservados con precisión, lo que confirma que el modelo fue capaz de capturar las características más esenciales de los datos en el espacio latente. Aunque las imágenes reconstruidas presentaron un leve suavizado, esto es consistente con el comportamiento esperado de los autoencoders, ya que el proceso de compresión tiende a eliminar detalles irrelevantes o ruido visual.

Finalmente, la visualización del espacio latente mediante la técnica \textit{t-SNE} permitió observar cómo el modelo organizó internamente las representaciones de los dígitos. A pesar de no haber sido entrenado con etiquetas, el autoencoder generó agrupamientos distinguibles para la mayoría de las clases, revelando que la codificación aprendida conserva información semántica significativa sobre las diferencias entre los números. Los solapamientos observados entre ciertas clases se atribuyen a similitudes visuales inherentes y a la limitación de representar un espacio latente complejo en dos dimensiones.



% --- BIBLIOGRAFÍA ---
% (Aquí es donde configurarías tu bibliografía, usualmente con .bib)
% --- BIBLIOGRAFÍA ---
% 1. Define el estilo de las referencias (IEEEtran es el estándar)
\bibliographystyle{IEEEtran}

% 2. Llama al archivo .bib (sin la extensión .bib)
\bibliography{referencias}

\end{document}